
\chapwithtoc{Závěr}
V rámci práce jsem implementoval hru na motivy Fantom staré Prahy a dva druhy umělých agentů a popsal základy teorie her a některé jejich aplikace. Popis dvou souvisejících bakalářských prací a velmi aktuálního článku mi poskytl zajímavé poznatky a vhled do řešení daného problému.

Implementoval jsem hru v herním engine Unity. Součástí práce je i hra samotná a její popis. Pro vývoj hry bylo potřebné získat znalosti o herním engine Unity a aplikovat je. 

Hlavním cílem této práce bylo implementovat algoritmus umělé inteligence do hry. V rámci práce jsem implementoval dva druhy umělých agentů. První využívá heuristiky vymyšlené na základě domény hry, druhý naopak využívá Monte Carlo techniky v algoritmu Information Set Monte Carlo Tree Search (ISMCTS), které přímo nevyužívají žádné znalosti domény. 

Pro otestování funkčnosti a úrovně umělých inteligencí jsem připravil menší turnaj. V prvním kole se utkala umělá inteligence využívající heuristiku proti náhodnému hráči. Heuristika vyhrála v obou případech, jak za Fantoma, tak za detektivy. Ve druhém kole se vítěz prvního kola utkal s~hráčem používajícím ISMCTS. Výsledky byly smíšené, za Fantoma vyhrál hráč s~ISMCTS, za detektivy naopak heuristika. Celý turnaj probíhal na třech různých mapách, v každém kole se provedlo 50 her pro obě možnosti umělých inteligencí.

Z výsledků experimentů vidíme, že umělí hráči hrají se snahou vyhrát. Díky výhrám na obou stranách můžeme usoudit, že zvolené mapy a parametry umožňují vyhrát jak Fantomovi, tak detektivům. Výherce je hráč s lepší strategií vůči jeho oponentovi. Algoritmus ISMCTS, který je nejen složitý, ale i aktuální, se prokázal být funkčním.

Implementace hry nabízí možnost za Fantoma i detektivy zvolit reálného i umělého hráče, umožňuje tedy hrát lidským hráčům proti sobě, ale i proti různě silným soupeřům umělé inteligence.