\chapter{Programátorská
dokumentace}\label{programuxe1torskuxe1-dokumentace}

\section{Graf}\label{graf}

Třída ``GameGraphScript'' je klíčovým prvkem herního prostředí,
zajišťujícím řízení interakcí mezi herním grafem a jeho uzly. Zodpovídá
za generování grafu na základě nových nebo existujících dat, přiřazování
identifikátorů uzlů, jejich vizuální reprezentaci a generování hran mezi
nimi. Umožňuje ukládání dat grafu a nabízí funkci hledání nejkratší
cesty mezi uzly (BFS).

\subsection{Vrchol}\label{vrchol}

Třída ``GameNodeScript'' je klíčovým prvkem herního prostředí,
zodpovědným za reprezentaci uzlů v herním grafu, jejich vizuální
zobrazení a interakci. Umožňuje nastavovat ID, pozici, spojené uzly a
transportní typy, a dále provádět ověřování spojení, manipulovat se
zvýrazňovačem a volat logiku na základě interakce. Struktura
``NodeData'' slouží k ukládání informací o uzlech.

\subsection{Hrana}\label{hrana}

Třída ``GameEdgeScript'' pouze drží texturu, kterou LineRenderer
vykreslí.

\section{Herní logika}\label{hernuxed-logika}

Třída ``GameLogicScript'' představuje jádro herní logiky a spravuje
průběh hry, interakce mezi hráči a grafem hry. Obsahuje odkazy na různé
herní komponenty, včetně grafu hry, správce historie Fantoma, ovladače
kamery, vykreslovače figurek, textu označujícího tah a panelu konce hry.
Třída řídí hráče a tahy, nastavuje viditelnost hráčů a tokenů,
implementuje vizuální efekty a animace tahu hráčů, spravuje informace o
poloze hráčů a tokeny, kontroluje podmínky ukončení hry a zajišťuje
hladký průběh herní smyčky. Kromě toho obsahuje metody pro správu konce
hry, zobrazuje konečný panel a ukládá statistiky. Třída také zahrnuje
utility pro správu cest k souborům.

\section{Config}\label{config}

Struktura ``Config'' zahrnuje několik vnořených struktur pro jednotlivé
konfigurační sekce. Každá konfigurační sekce obsahuje několik proměnných
s hodnotami nastavení. Metody BasicConfig, LoadConfig, LoadConfigOrBasic
a SeeStructure umožňují manipulaci s konfiguracemi.

\section{Logy}\label{logy}

Rozhraní ``IGameLogger'' definuje metody pro logování informací o
průběhu hry, chybách a dalších událostech.

Třída ``EmptyLogger'' implementuje rozhraní ``IGameLogger'' a všechny
logy zahazuje, tj. neprovádí žádnou skutečnou operaci.

Třída ``FileLogger'' také implementuje rozhraní ``IGameLogger'' a slouží
k logování informací do souboru. Konstruktor třídy ``FileLogger''
inicializuje logovací mechanismus a vytváří soubor, pokud neexistuje.
Metody LogTurn, LogError a LogInfo slouží k zaznamenávání různých typů
zpráv do souboru. Metody WriteLine a Write slouží k zápisu zpráv do
souboru, s ohledem na vláknovou bezpečnost pomocí uzamčení (lock).

\section{Ukládání map}\label{ukluxe1duxe1nuxed-map}

Třída ``MapData'' je odvozena od třídy ScriptableObject a slouží k
uchování dat mapy. Obsahuje informace o JSON datovém řetězci a zda je
mapa vybrána.

Struktura ``NodeDataHolder'' obsahuje informace o uzlech na mapě, včetně
jejich dopravních spojení a pozic.

Struktura ``StringPosition'' slouží k uchování pozice jako řetězce pro
JSON serializaci.

Struktura ``EnumIntPair'' uchovává páry klíč-hodnota, kde klíč je
celočíselný identifikátor a hodnota je seznam celočíselných hodnot.

Struktura ``EnumDictionary'' uchovává slovník, který mapuje klíče na
seznam hodnot.

Třída ``NodeConverter'' obsahuje metody pro konverzi instancí třídy
GameNodeScript na JSON. Tato třída umožňuje převést uzly mapy a seznam
uzlů na JSON formát.

\section{Generování map}\label{generovuxe1nuxed-map}

Rozhraní IGraphGenerator definuje metody pro generování uzlů a načítání
konfigurace.

Třída SimpleGraphGenerator implementuje rozhraní IGraphGenerator a
provádí generování uzlů a hran v grafu na základě zadané konfigurace.
Tato třída obsahuje metody pro generování uzlů, včetně zpracování
existujících uzlů a vytváření hran mezi nimi. Taktéž zde jsou
implementovány metody pro zpracování konfigurace a vytváření spojení
mezi uzly.

\section{Hráč}\label{hruxe1ux10d}

\subsection{AI}\label{ai}

Rozhraní ``IFantom'' a ``IDetectives'' slouží jako kontrakty pro hráče
Fantoma a detektivů. Obě tato rozhraní dědí od základního rozhraní
``IPlayerBase'', které obsahuje metody pro správu tahů, nastavení
dostupných transportních tokenů a reakci na tahy protivníka.

\subsection{Reálný hráč}\label{reuxe1lnuxfd-hruxe1ux10d}

Třída ``RealBase'' je abstraktní třídou pro reálné hráče a obsahuje
logiku pro reakci na kliknutí na herní uzel.

\subsection{Struktura Move}\label{struktura-move}

Struktura ``Move'' reprezentuje tah hráče a obsahuje informace o pozici
a druhu dopravního prostředku.

\subsection{Hledání hráčů}\label{hleduxe1nuxed-hruxe1ux10dux16f}

Třída ``PlayerFinder'' je utilitní třídou, která umožňuje nalézt typy
implementující dané rozhraní v určitém assembly. Obsahuje metodu
``GetTypesWithInterface'', která vrací seznam typů, které implementují
specifikované rozhraní ``T''. Dále obsahuje metodu
``GetClassNamesFromTypes'', která z kolekce typů získá seznam názvů
těchto typů.

Třída ``TypeLoaderExtensions'' obsahuje rozšíření pro načítání typů z
assembly. Metoda ``GetLoadableTypes'' umožňuje získat načitatelné typy z
dané assembly a zároveň zachytává výjimky, které mohou nastat během
načítání typů.

\section{Vykreslování postav}\label{vykreslovuxe1nuxed-postav}

Třída ``FigureRendererScript'' se stará o vykreslování a pohyb herních
figurek (detektivů a Fantoma). Figurky jsou reprezentovány pomocí
sprite a barev. Třída obsahuje metody pro inicializaci a přípravu
figurek, a také pro animovaný pohyb figurek na cílovou pozici.

Dále třída ``ColorUtility'' slouží pro operace spojené s barvami,
například pro výpočet podobnosti barev a zjišťování, zda je barva příliš
světlá pro bílý text.


\chapter{Uživatelská dokumentace}\label{uux17eivatelskuxe1-dokumentace}





